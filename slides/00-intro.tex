%% -*- coding: utf-8 -*-
\documentclass{beamer}

\input{common}
%%% frontmatter
%% -*- coding: utf-8 -*-

\title{Functional Programming}
\subtitle{Introduction}

\author[Peter Thiemann]{Prof. Dr. Peter Thiemann}
\institute[Univ. Freiburg]{Albert-Ludwigs-Universität Freiburg, Germany}
\date{WS 2024/25}


\usepackage{tikz}


\begin{document}

\begin{frame}
  \titlepage
\end{frame}

\begin{frame}[fragile]
  \frametitle{Coordinates}
  \begin{itemize}
  \item \textbf{Course hours:}  Tu 14-16
  \item \textbf{Course room:} SR 106-04-007 Videokonferenz
  \item \textbf{Staff:} Prof.\ Dr.\ Peter Thiemann\\
\begin{alltt}
Gebäude 079, Raum 00-015
Telefon: 0761 203 -8051/-8247
E-mail: thiemann@cs.uni-freiburg.de
Web: \href{http://www.informatik.uni-freiburg.de/~thiemann}{http://www.informatik.uni-freiburg.de/~thiemann}
\end{alltt}
\item  \textbf{Staff:} Hannes Saffrich\\
\begin{alltt}
E-mail: saffrich@cs.uni-freiburg.de
\end{alltt}
  \item\textbf{Homepage:}\\ \footnotesize
    \href{https://github.com/proglang/FunctionalProgramming}{
      https://github.com/proglang/FunctionalProgramming}
    \\
    \href{https://proglang.informatik.uni-freiburg.de/teaching/functional-programming/2024/}{
      https://proglang.informatik.uni-freiburg.de/teaching/functional-programming/2024/}
  \end{itemize}
\end{frame}
%----------------------------------------------------------------------
\begin{frame}
  \frametitle{Administrivia}
  \begin{block}{Lecture \~{}90 minutes/week}
    \begin{itemize}
    \item lecture videos will be made available on the webpage
    \item presence with live recording (zoom)
    \end{itemize}
  \end{block}
  \begin{block}{Exercise  \~{}90 minutes exercise/week}
    \begin{itemize}
    \item Friday 14-16, SR 00 031, building 051
    \item exercise questions modality TBA
    \begin{itemize}
    \item discussed in next available exercise session
    \item no need to hand in exercises during the semester
    \end{itemize}
  \end{itemize}
\end{block}
\begin{block}{Final exam}
  \begin{itemize}
  \item written exam in the computer pool
  \item comprising theory questions and small programming tasks
  \end{itemize}
\end{block}
\end{frame}

%----------------------------------------------------------------------
% \begin{frame}
%   \frametitle{Administrivia - Exercise}
%   \begin{itemize}
%   \item \~{}90 minutes exercise/week
%     \begin{itemize}
%     \item mode TBD
%     \end{itemize}
%   \item Exercise questions available on Wednesdays
%     \begin{itemize}
%     \item discussed in next available exercise session
%     \item no need to hand in exercises during the semester
%     \item final exam will be a written exam (comprising theory
%       questions and small programming tasks)
%     \end{itemize}
%   \end{itemize}
% \end{frame}


%----------------------------------------------------------------------

\begin{frame}
  \frametitle{Contents}
  \begin{itemize}
  \item Basics of functional programming using Haskell
  \item Theoretical background
  \item Writing Haskell programs
  \item Using Haskell libraries and development tools
  \item Your first Haskell project
  \end{itemize}
\end{frame}


%----------------------------------------------------------------------
\begin{frame}
  \frametitle{What is Haskell?}
  \begin{quotation}
    In September of 1987 a meeting was held at the conference on
    Functional Programming Languages and Computer Architecture in
    Portland, Oregon, to discuss an unfortunate situation in the
    functional programming community: there had come into being more
    than a dozen non-strict, purely functional programming languages,
    all similar in expressive power and semantic underpinnings. There
    was a strong consensus at this meeting that more widespread use
    of this class of functional languages was being hampered by the
    lack of a common language. It was decided that a committee should
    be formed to design such a language, providing faster
    communication of new ideas, a stable foundation for real
    applications development, and a vehicle through which others would 
    be encouraged to use functional languages. 
  \end{quotation}
  {\tiny From ``History of Haskell''}
\end{frame}

%----------------------------------------------------------------------
\begin{frame}
  \frametitle{What is Functional Programming?}
  \begin{block}<+->{A different approach to programming}
    \begin{LARGE}
      \begin{center}
        Functions and values
        \\[3ex]
        rather than
        \\[3ex]
        Assignments and pointers
      \end{center}
    \end{LARGE}
  \end{block}
  \begin{alertblock}<+->{It will make you a better programmer}
    
  \end{alertblock}
\end{frame}
%----------------------------------------------------------------------
\begin{frame}
  \frametitle{Why Haskell?}
  \begin{itemize}
  \item Haskell is a very high-level language
    \\\hfill\textcolor{gray}{(many details taken care
    of automatically)}. 
  \item Haskell is expressive and concise
    \\\hfill\textcolor{gray}{(can achieve a lot with a
    little effort)}. 
  \item Haskell is good at handling complex data and combining
    components. 
  \item Haskell is a high-productivity language \\\hfill\textcolor{gray}{ (prioritizes programmer-time over computer-time)}
  \end{itemize}
\end{frame}
%----------------------------------------------------------------------

\begin{frame}[fragile]
  \frametitle{Functional vs Imperative Programming: Variables}
  \begin{block}<+->{Functional (Haskell)}
\begin{verbatim}
x :: Int
x = 5
\end{verbatim}
    Variable \texttt{x} has value \texttt{5} forever
  \end{block}
  \begin{block}<+->{Imperative (Java / C)}
\begin{verbatim}
int x = 5;
...
x = x+1;
\end{verbatim}
    Variable \texttt{x} can change its content over time
  \end{block}
\end{frame}

\begin{frame}[fragile]
  \frametitle{Functional vs Imperative Programming: Functions}
  \begin{block}<+->{Functional (Haskell)}
\begin{verbatim}
f :: Int -> Int -> Int
f x y = 2*x + y

f 42 16 // always 100
\end{verbatim}
    Return value of a function \textbf{only} depends on its inputs
  \end{block}
  \begin{block}<+->{Imperative (Java)}
\begin{verbatim}
boolean flag;
static int f (int x, int y) {
  return flag ? 2*x + y , 2*x - y;
}

int z = f (42, 16); // who knows?
\end{verbatim}
    Return value  depends on non-local variable \texttt{flag}
  \end{block}
\end{frame}

\begin{frame}
  \frametitle{Equational Reasoning and Referential Transparency}
  \begin{block}<+->{Haskell}
    \begin{enumerate}
    \item Variables have a fixed value
    \item Functions have a fixed meaning that does not depend on
      context\\
      (Actually, the same as item~1!)
    \end{enumerate}
  \end{block}
  \begin{block}{Consequence: Equational Reasoning}
    \begin{itemize}
    \item Every (sub-) expression in a program can be replaced with
      its value and
    \item every function call can be replaced with its definition
    \item \dots{} without  without changing the meaning of the
      program!
    \item (may change the running time, though)
    \end{itemize}
  \end{block}
\end{frame}

\begin{frame}[fragile]
  \frametitle{Functional vs Imperative Programming: Laziness}
  \begin{block}<+->{Haskell}
\begin{verbatim}
x = expensiveComputation
g anotherExpensiveComputation
\end{verbatim}
    \begin{itemize}
    \item The expensive computation will only happen if \texttt{x} is
      ever used.
    \item Another expensive computation will only happen if \texttt{g}
      uses its argument.
    \end{itemize}
  \end{block}
  \begin{block}<+->{Java}
\begin{verbatim}
int x = expensiveComputation;
g (anotherExpensiveComputation)
\end{verbatim}
    \begin{itemize}
    \item Both expensive computations will happen anyway.
    \item Laziness can be simulated, but it's complex!
  \end{itemize}
  \end{block}
\end{frame}

%----------------------------------------------------------------------
\begin{frame}
  \frametitle{Many more features that make programs more concise}
  \begin{itemize}
  \item Static typing
  \item Algebraic datatypes
  \item Polymorphic types
  \item Parametric overloading
  \item Type inference
  \item Monads \& friends (for IO, concurrency, \dots)
  \item Comprehensions
  \item Concurrency
  \item Metaprogramming
  % \item Domain specific languages
  \item \dots
  \end{itemize}
\end{frame}
%----------------------------------------------------------------------
\begin{frame}
  \frametitle{References}
  \begin{itemize}
  \item Paper by the original developers of Haskell in the conference on History of
    Programming Languages (HOPL III):\\
    \href{http://dl.acm.org/citation.cfm?id=1238856}{A History of Haskell: Being Lazy with Class}
  \item The Haskell home page: \url{http://www.haskell.org}
  \item Haskell libraries repository:
    \url{https://hackage.haskell.org/}
  \item Haskell Tool Stack: \url{https://docs.haskellstack.org/en/stable/README/}
  \end{itemize}
\end{frame}

%----------------------------------------------------------------------

\begin{frame}
  % \frametitle{Questions?}
  \begin{center}
    \tikz{\node[scale=15] at (0,0){?};}
  \end{center}
\end{frame}



\end{document}

%%% Local Variables: 
%%% mode: latex
%%% TeX-master: t
%%% End: 
