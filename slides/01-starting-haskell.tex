%% -*- coding: utf-8 -*-
\documentclass{beamer}

\input{common}
%%% frontmatter
%% -*- coding: utf-8 -*-

\title{Functional Programming}
\subtitle{Introduction}

\author[Peter Thiemann]{Prof. Dr. Peter Thiemann}
\institute[Univ. Freiburg]{Albert-Ludwigs-Universität Freiburg, Germany}
\date{WS 2024/25}


\subtitle{Starting Haskell}
\usepackage{tikz}


\begin{document}

\begin{frame}
  \titlepage
\end{frame}
%----------------------------------------------------------------------

\begin{frame}
  \begin{Huge}
    \begin{center}
      Let's get started!
    \end{center}
  \end{Huge}
\end{frame}

%----------------------------------------------------------------------

\begin{frame}[fragile]
  \frametitle{Haskell Demo}
  \begin{itemize}
  \item  Let's say we want to buy a game in the USA and we have to
    convert its price from USD to EUR
  \item  A \textbf{definition} gives a name to a value
  \item Names are  case-sensitive, must start with lowercase letter
  \item Definitions are  put in a text file ending in \texttt{.hs}
  \end{itemize}
  \begin{block}{Examples.hs}
\begin{verbatim}
 dollarRate2018 = 1.18215  
 dollarRate2019 = 1.3671
 dollarRate2022 = 0.98546541
 dollarRate = 0.91691801
\end{verbatim}
  \end{block}
\end{frame}
\begin{frame}[fragile]
  \frametitle{Using the definition}
  \begin{itemize}
  \item Start the Haskell interpreter  GHCi\\
    \texttt{> stack ghci}
\begin{verbatim}
Configuring GHCi with the following packages: 
GHCi, version 9.0.2: http://www.haskell.org/ghc/  :? for help
Loaded GHCi configuration from /private/var/folders/t4/sknv38zx3qv7c0lm9xj9nhrc0000gn/T/ghci33367/ghci-script
Prelude> 
\end{verbatim}
  \item Load the file\\
    \texttt{Prelude> :l Examples.hs}
\begin{verbatim}
[1 of 1] Compiling Main             ( Examples.hs, interpreted )
Ok, modules loaded: Main.
*Main> 
\end{verbatim}
  \item Use the definition
\begin{verbatim}
*Main> dollarRate
0.91691801
*Main> 53 * dollarRate
48.596654529999995
\end{verbatim}
  \end{itemize}
\end{frame}
\begin{frame}[fragile]
  \frametitle{A function to convert EUR to USD}
\begin{block}{Examples.hs}
\begin{verbatim}
dollarRate = 0.91691801

-- |convert EUR to USD
usd euros = euros * dollarRate
\end{verbatim}
  \end{block}
  \begin{itemize}
  \item line starting with \texttt{--}: comment
  \item \texttt{usd}: function name (defined)
  \item \texttt{euros}:  argument name (defined)
  \item \texttt{euros * dollarRate}: expression to compute the result
  \end{itemize}
\end{frame}
\begin{frame}[fragile]
  \frametitle{Using the function}
  \begin{itemize}
  \item load into GHCi
    \begin{itemize}
    \item as before or
    \item use \texttt{:r} to reload
    \end{itemize}
  \end{itemize}
\begin{verbatim}
*Main> usd 1
0.91691801
*Main> usd 73
66.93501472999999
\end{verbatim}
\end{frame}
\begin{frame}[fragile]
  \frametitle{Converting back}
Write a function \texttt{euro} that converts back from USD to EUR!
\begin{verbatim}
*Main> euro (usd 73)
73.0
*Main> euro (usd 1)
1.0
*Main> usd (euro 100)
100.0
\end{verbatim}
\begin{alertblock}<2->{Your turn}

\end{alertblock}
\end{frame}

\begin{frame}[fragile]
  \frametitle{Testing properties}
  \framesubtitle{Is this function correct?}
  \begin{block}<+->{A reasonable property of \texttt{euro} and \texttt{usd}}
\begin{verbatim}
prop_EuroUSD x = euro (usd x) == x
\end{verbatim}
    \texttt{==} is the equality operator
\begin{verbatim}
*Main> prop_EuroUSD 79
True
*Main> prop_EuroUSD 1
True
\end{verbatim}
  \end{block}
  \begin{alertblock}<+->{Does it hold in general?}
    
  \end{alertblock}
\end{frame}
\begin{frame}[fragile]
  \frametitle{Aside: Testing by Writing Properties}
  \begin{block}{Convention}
    Function names beginning with
    \verb|prop_| are properties we expect to be True 
  \end{block}
  \begin{block}{Writing properties in a file}
    \begin{itemize}
    \item Tells us how functions should behave 
    \item Tells us what has been tested 
    \item Lets us repeat tests after changing a definition 
    \end{itemize}
  \end{block}
\end{frame}

\begin{frame}[fragile]
  \frametitle{Testing}
  \begin{block}<+->{At the beginning of Examples.hs}
\begin{verbatim}
import Test.QuickCheck
\end{verbatim}
    A widely used Haskell library for automatic random testing
  \end{block}
  \begin{block}<+->{May need to install it first \dots}
\begin{verbatim}
stack install QuickCheck
\end{verbatim}
  \end{block}
\end{frame}
\begin{frame}[fragile]
  \frametitle{Running tests}
\begin{verbatim}
*Main> quickCheck prop_EuroUSD
*** Failed! Falsifiable (after 10 tests and 1 shrink): 
7.0
\end{verbatim}
  \begin{itemize}
  \item Runs 100 randomly chosen tests
  \item Result: The property is wrong!
  \item It fails for input 7.0
  \end{itemize}
  \begin{alertblock}<2->{Check what happens for 7.0!}
    
  \end{alertblock}
\end{frame}
\begin{frame}[fragile]
  \frametitle{What happens for 7.0}
\begin{verbatim}
*Main> usd 1.1
1.0086098110000001
*Main> euro 1.0086098110000001
1.1000000000000003
\end{verbatim}
\end{frame}
\begin{frame}[fragile]
  \frametitle{The Problem: Floating Point Arithmetic}
  \begin{itemize}
  \item  There is a tiny difference between the initial and final values 
\begin{verbatim}
*Main> euro (usd 1.1) - 1.1
2.220446049250313e-16
\end{verbatim}
  \item Calculations are only performed to about 15 significant
    figures 
  \item  The property is wrong! 
  \end{itemize}
\end{frame}
\begin{frame}
  \frametitle{Fixing the problem}
  \begin{itemize}
  \item NEVER use equality with floating point numbers!
  \item  The result should be \emph{nearly} the same 
  \item  The difference should be small – smaller than 10E-15
  \end{itemize}
\end{frame}
\begin{frame}[fragile]
  \frametitle{Comparing Values}
\begin{verbatim}
*Main> 2<3
True
*Main> 3<2
False
\end{verbatim}
\end{frame}
\begin{frame}[fragile]
  \frametitle{Defining ``Nearly Equal''}
  \begin{itemize}
  \item Can define new operators with names made up of symbols
  \end{itemize}
\begin{block}{In Examples.hs}
\begin{verbatim}
x ~== y = x - y < 10e-15
\end{verbatim}
  \end{block}
\begin{verbatim}
*Main> 3 ~== 3.0000001
True
*Main> 3 ~== 4
True
\end{verbatim}
\end{frame}
\begin{frame}[fragile,fragile]
  \frametitle{Defining ``Nearly Equal''}
  \begin{itemize}
  \item Can define new operators with names made up of symbols
  \end{itemize}
\begin{block}{In Examples.hs}
\begin{verbatim}
x ~== y = abs(x - y) < 10e-15 * abs x
\end{verbatim}
  \end{block}
\begin{verbatim}
*Main> 3 ~== 3.0000001
True
*Main> 3 ~== 4
True
\end{verbatim}
\end{frame}
\begin{frame}[fragile]
  \frametitle{Fixing the property}
\begin{block}{In Examples.hs}
\begin{verbatim}
prop_EuroUSD' x = euro (usd x) ~== x
\end{verbatim}
  \end{block}
\begin{verbatim}
*Main> prop_EuroUSD' 3
True
*Main> prop_EuroUSD' 56
True
*Main> prop_EuroUSD' 7
True
\end{verbatim}
\end{frame}

\begin{frame}[fragile]
  \frametitle{Name the price}
  \begin{block}<+->{Let's define a name for the price of the game we want
      in Examples.hs}
\begin{verbatim}
price = 79
\end{verbatim}
  \end{block}
  \begin{alertblock}<+->{After reload: Ouch!}
\begin{verbatim}
*Main> euro price

<interactive>:57:6:
    Couldn't match expected type `Double' with actual type `Integer'
    In the first argument of `euro', namely `price'
    In the expression: euro price
    In an equation for `it': it = euro price
\end{verbatim}
  \end{alertblock}
\end{frame}
\begin{frame}[fragile]
  \frametitle{Every Value has a type}
  The \texttt{:i} command prints information about a name
\begin{verbatim}
*Main> :i price
price :: Integer
  	-- Defined at ...
*Main> :i dollarRate
dollarRate :: Double
  	-- Defined at ...
\end{verbatim}
\end{frame}
\begin{frame}[fragile]
  \frametitle{More types}
\begin{verbatim}
*Main> :i True
data Bool = ... | True 	-- Defined in `GHC.Types'
*Main> :i False
data Bool = False | ... 	-- Defined in `GHC.Types'
*Main> :i euro
euro :: Double -> Double
  	-- Defined at...
*Main> :i prop_EuroUSD'
prop_EuroUSD' :: Double -> Bool
  	-- Defined at...
\end{verbatim}
  \begin{itemize}
  \item \texttt{True} and \texttt{False} are \textbf{data constructors}
  \end{itemize}
\end{frame}
\begin{frame}[fragile]
  \frametitle{Types matter}
  \begin{itemize}
  \item Types determine how computations are performed
  \item A type annotation specifies which type to use
  \end{itemize}
\begin{verbatim}
*Main> 123456789*123456789 :: Double
1.524157875019052e16
*Main> 123456789*123456789 :: Integer
15241578750190521
\end{verbatim}
  \begin{itemize}
  \item \texttt{Double}: double precision floating point 
  \item  \texttt{Integer}: exact computation
  \item  GHCi must know the type of each expression before computing it.
  \end{itemize}
\end{frame}
\begin{frame}
  \frametitle{Type inference and type checking}
  \begin{itemize}
  \item An algorithm \textbf{infers} (works out) the type of every
    expression
  \item It finds the ``best'' type for each expression
  \item Checks that all types match --- before running the program
  \end{itemize}
\end{frame}
\begin{frame}[fragile]
  \frametitle{Our example}
\begin{verbatim}
*Main> :i price
price :: Integer
  	-- Defined at...
*Main> :i euro
euro :: Double -> Double
  	-- Defined at...
*Main> euro price

<interactive>:70:6:
    Couldn't match expected type `Double' with actual type `Integer'
    In the first argument of `euro', namely `price'
    In the expression: euro price
    In an equation for `it': it = euro price
\end{verbatim}
\end{frame}
\begin{frame}[fragile]
  \frametitle{Why did it work before?}
  \begin{itemize}
  \item Numeric literals are \textbf{overloaded}: they can be
    used with several types
  \item  Giving the number a name fixes its type
  \end{itemize}
\begin{verbatim}
*Main> euro 79
57.78655548240802
*Main> 79 :: Integer
79
*Main> 79 :: Double
79.0
*Main> price :: Integer
79
*Main> price :: Double

<interactive>:76:1:
    Couldn't match expected type `Double' with actual type `Integer'
    In the expression: price :: Double
    In an equation for `it': it = price :: Double
\end{verbatim}
\end{frame}
\begin{frame}[fragile]
  \frametitle{Fixing the problem/1}
  A definition can be given a \textbf{type signature} which specifies
  its type
\begin{block}{In Examples.hs}
\begin{lstlisting}
-- |price of the game in USD
price' :: Double
price' = 79
\end{lstlisting}
  \end{block}
\begin{verbatim}
*Main> :i price'
price' :: Double
  	-- Defined at...
*Main> euro price'
72.43652279
\end{verbatim}
\end{frame}
\begin{frame}[fragile]
  \frametitle{Fixing the problem/2}
  Reintroduce the overloading using function \texttt{fromInteger} (a
  type cast), which converts to any numeral type
\begin{verbatim}
*Main> :i price
price :: Integer
  	-- Defined at...
*Main> euro (fromInteger price)
72.43652279
\end{verbatim}
\end{frame}

%----------------------------------------------------------------------

\begin{frame}
  \frametitle{Questions?}
  \begin{center}
    \tikz{\node[scale=15] at (0,0){?};}
  \end{center}
\end{frame}


\end{document}

%%% Local Variables: 
%%% mode: latex
%%% TeX-master: t
%%% End: 
