%% -*- coding: utf-8 -*-
\documentclass{beamer}

\input{common}
%%% frontmatter
%% -*- coding: utf-8 -*-

\title{Functional Programming}
\subtitle{Introduction}

\author[Peter Thiemann]{Prof. Dr. Peter Thiemann}
\institute[Univ. Freiburg]{Albert-Ludwigs-Universität Freiburg, Germany}
\date{WS 2024/25}


\subtitle{Type Classes --- Overloading in Haskell}
\usepackage{tikz}


\begin{document}

\begin{frame}
  \titlepage
\end{frame}
%----------------------------------------------------------------------
\begin{frame}[fragile]
  \frametitle{Overloading}
  \begin{block}<+->{Remember previous classes?}
    We were able to use
    \begin{itemize}
    \item equality \texttt{==} and ordering \texttt{<} with many different types
    \item arithmetic operations with many different types
    \end{itemize}
  \end{block}
  \begin{block}<+->{Overloading}
    The \textbf{same operator} can be used to execute \textbf{different code} at \textbf{many different types}. 
  \end{block}
  \onslide<+->{Contrast with }
  \begin{block}<+->{Parametric polymorphism}
    The \textbf{same code} can execute at \textbf{many different types}.
  \end{block}
\end{frame}
%----------------------------------------------------------------------
\begin{frame}[fragile]
  \frametitle{Haskell integrates overloading with polymorphism}
  \begin{block}<+->{Constrained polymorphism}
    \begin{itemize}
    \item Some functions work on parametric types, but are constrained to specific instances
    \item Types contain type variables and \textbf{constraints}
    \end{itemize}
  \end{block}
  \begin{block}<+->{Examples}
\begin{lstlisting}
-- elem x xs : is x an element of list xs?
-- type a must have equality
elem :: Eq a => a -> [a] -> Bool
-- insert x xs : insert x into sorted list xs
-- type a must have comparison
insert :: Ord a => a -> [a] -> [a]
-- square x : compute the square of x
-- type a has numeric operations
square :: Num a => a -> a
\end{lstlisting}
  \end{block}
\end{frame}
%----------------------------------------------------------------------
\begin{frame}[fragile]
  \frametitle{Type classes}
  \begin{itemize}
  \item Each constraint mentions a \textbf{type class}\\
    like \texttt{Eq}, \texttt{Ord}, \texttt{Num}, \dots
  \item A type class specifies a set of operations for a type\\
    e.g.\ \texttt{Eq} requires \texttt{==} and \texttt{/=}
  \item Type classes form a hierarchy\\
    e.g.\ \texttt{Ord a => Eq a}
  \item Many classes are predefined, but you can roll your own
  \end{itemize}
\end{frame}
%----------------------------------------------------------------------
\begin{frame}[fragile]
  \frametitle{Classes and Instances}
  \begin{itemize}
  \item A \emph{class declaration} specifies a signature (i.e., the class members and their types)
\begin{lstlisting}
class Num a where
  (+), (*), (-) :: a -> a -> a
  negate, abs, signum :: a -> a
  fromInteger :: Integer -> a
\end{lstlisting}
  \item An \emph{instance declaration} specifies that a type belongs to a class by giving definitions for all class members
\begin{lstlisting}
instance Num Int where ...
instance Num Integer where ...
instance Num Double where ...
instance Num Float where ...
\end{lstlisting}
  \item This info can be obtained from GHCI by
\begin{lstlisting}
:i Num 
\end{lstlisting}
  \end{itemize}
\end{frame}
%----------------------------------------------------------------------
\begin{frame}[fragile]
  \frametitle{Equality}
  \begin{block}<+->{The type class \texttt{Eq}}
\begin{lstlisting}
class Eq a where
  (==), (/=) :: a -> a -> Bool
  x /= y = not (x == y)         -- default definition
\end{lstlisting}
    An instance must only provide \texttt{(==)}.
  \end{block}
  \begin{block}<+->{Instances of \texttt{Eq}}
\begin{lstlisting}
instance Eq Int -- Defined in 'GHC.Classes'
instance Eq Float -- Defined in 'GHC.Classes'
instance Eq Double -- Defined in 'GHC.Classes'
instance Eq Char -- Defined in 'GHC.Classes'
instance Eq Bool -- Defined in 'GHC.Classes'
{- and many more -}
\end{lstlisting}
  \end{block}
  \begin{exampleblock}<+->{Tacit assumption}
    Equality is a congruence relation.
  \end{exampleblock}
\end{frame}
%----------------------------------------------------------------------
\begin{frame}[fragile]
  \frametitle{Defining \texttt{Eq} for pairs}
  \begin{block}<+->{When are two pairs equal?}
  \end{block}
  \begin{block}<+->{Solution}
\begin{lstlisting}
instance (Eq a, Eq b) => Eq (a, b) where
  (a1, b1) == (a2, b2) = a1 == a2 && b1 == b2
\end{lstlisting}
  \end{block}
  \begin{block}<+->{Is this definition recursive?}
  \end{block}
  \begin{alertblock}<+->{YES: on types, NO: on values}
  \end{alertblock}
\end{frame}
%----------------------------------------------------------------------
\begin{frame}[fragile]
  \frametitle{Defining \texttt{Eq} for lists}
  \begin{block}<+->{When are two lists equal?}
  \end{block}
  \begin{block}<+->{Solution}
\begin{lstlisting}
instance Eq a => Eq [a] where
  [] == [] = True
  (x:xs) == (y:ys) = x == y && xs == ys
  _ == _ = False
\end{lstlisting}
  \end{block}
  \begin{block}<+->{Is this definition recursive?}
  \end{block}
  \begin{alertblock}<+->{YES: on types, YES: on values}
    The equality \texttt{xs == ys}.
  \end{alertblock}
\end{frame}
%----------------------------------------------------------------------
\begin{frame}[fragile]
  \frametitle{Handwriting vs deriving an instance}
  \begin{block}<+->{Remember the Hearts game}
\begin{lstlisting}
data Color = Black | Red
  deriving (Show)
\end{lstlisting}
  \end{block}
  \begin{block}<+->{Define your own equality}
\begin{lstlisting}
instance Eq Color where
  Black == Black = True
  Red == Red = True
  _ == _ = False
\end{lstlisting}
  \end{block}
  \begin{block}<+->{Same result as deriving \texttt{Eq}}
\begin{lstlisting}
data Color = Black | Red
  deriving (Show, Eq)
\end{lstlisting}
  \end{block}
\end{frame}
%----------------------------------------------------------------------
\begin{frame}[fragile]
  \frametitle{Further useful classes}
  \begin{block}<+->{Show and Read}
\begin{lstlisting}
class Show a where
  show :: a -> String
  {- ... -}

class Read a where
  read :: String -> a
  {- ... -}
\end{lstlisting}
    \begin{itemize}
    \item Predefined for most built-in types
    \item Derivable for most datatype definitions
    \end{itemize}
  \end{block}
\end{frame}
%----------------------------------------------------------------------
\begin{frame}[fragile]
  \frametitle{The \texttt{Ord} class (derivable)}
\begin{lstlisting}
class Eq a => Ord a where
  compare :: a -> a -> Ordering
  (<) :: a -> a -> Bool
  (<=) :: a -> a -> Bool
  (>) :: a -> a -> Bool
  (>=) :: a -> a -> Bool
  max :: a -> a -> a
  min :: a -> a -> a

data Ordering = LT | EQ | GT 	-- Defined in 'GHC.Types'
\end{lstlisting}
\end{frame}
%----------------------------------------------------------------------
\begin{frame}[fragile]
  \frametitle{More classes for you to investigate}
  \begin{itemize}
  \item \texttt{Enum} (derivable)
  \item \texttt{Bounded} (derivable)
  \end{itemize}
\begin{lstlisting}
\end{lstlisting}
\end{frame}
%----------------------------------------------------------------------
\begin{frame}[fragile]
  \frametitle{Ambiguity}
\onslide<+->{Some combinations of overloaded functions can lead to ambiguity}
\begin{lstlisting}
f x = read (show x)
g x = show (read x)
\end{lstlisting}
\begin{block}<+->{What are types of \texttt{f} and \texttt{g}?}
\end{block}
\begin{block}<+->{Solution}
\begin{lstlisting}
f :: (Read a, Show b) => b -> a
g :: String -> String
\end{lstlisting}
\end{block}
\end{frame}
%----------------------------------------------------------------------
\begin{frame}[fragile]
  \frametitle{Further pitfalls / features}
  \begin{itemize}
  \item Definitions without arguments and without type signatures are not overloaded (monomorphism restriction)
  \item Numeric literals are overloaded at type \texttt{Num a => a}
  \item Haskell has a \textbf{defaulting} mechanism that resolves violations of the monomorphism restriction 
  \item Caveat: GHCi behaves differently than code in a file
  \end{itemize}
\end{frame}
%----------------------------------------------------------------------
%----------------------------------------------------------------------
\begin{frame}
  \frametitle{Wrapup}
  \begin{alertblock}{Type classes}
  \begin{itemize}
  \item provide a signature for an abstract data type
  \item instances provide implementations at unrelated types
  \item many classes are predefined and derivable
  \item pervasively used in Haskell / some pitfalls
  \end{itemize}
  \end{alertblock}
\end{frame}

% \begin{frame}
%   \frametitle{Break Time --- Questions?}
%   \begin{center}
%     \tikz{\node[scale=15] at (0,0){?};}
%   \end{center}
% \end{frame}


\end{document}

%%% Local Variables: 
%%% mode: latex
%%% TeX-master: t
%%% End: 
