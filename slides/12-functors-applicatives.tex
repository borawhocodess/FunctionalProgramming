%% -*- coding: utf-8 -*-
\documentclass{beamer}

\input{common}
%%% frontmatter
%% -*- coding: utf-8 -*-

\title{Functional Programming}
\subtitle{Introduction}

\author[Peter Thiemann]{Prof. Dr. Peter Thiemann}
\institute[Univ. Freiburg]{Albert-Ludwigs-Universität Freiburg, Germany}
\date{WS 2024/25}


\subtitle
{Functors, Applicatives, and Parsers}

\begin{document}
\AtBeginSection[]
{
 \begin{frame}<beamer>
 \frametitle{Plan}
 \tableofcontents[currentsection]
 \end{frame}
}
\begin{frame}
  \titlepage
\end{frame}

\begin{frame}
  \frametitle{Introduction}
  \begin{itemize}
  \item Functors and applicatives are concepts from \alert{category
      theory}
  \item A very general and abstract theory about structures and maps between them
  \item So general that mathematicians call it ``general abstract
    nonsense''
  \item Yields very useful abstractions for functional programming
  \item After a brief review we specialize for Haskell
  \end{itemize}
\end{frame}

\section{Categories}
\begin{frame}
  \frametitle{(Small) Categories}
  \begin{block}{Definition (part 1)}
    A \alert{small category} is given by
    \begin{itemize}
    \item a set of \alert{objects},
    \item for each pair of objects $A,B$, a set $Hom(A,B)$ of
      \alert{arrows} (morphisms) between $A$ and $B$,
    \item for each pair of arrows $f \in Hom(A,B)$ and $g \in Hom (B,
      C)$ (for objects $A,B,C$), there is an arrow $(f;g) \in Hom (A,
      C)$, the \alert{composition} of $f$ and $g$ (alternativel, write
      $g\circ f$).
    \end{itemize}
    Moreover, the following laws are expected to hold
  \end{block}
\end{frame}
\begin{frame}
  \frametitle{(Small) Categories}
  \begin{block}{Definition (part 2: laws)}
    \begin{itemize}
    \item For each object $A$ there is a designated
      \alert{identity arrow} $i_A \in Hom (A,A) $ which behaves as an
      identity with respect to composition:
      \begin{itemize}
      \item for each $f \in Hom (A, B)$, $i_A;f = f$,
      \item for each $g \in Hom (B, A)$, $g; i_B = g$.
      \end{itemize}
    \item Composition of arrows is associative, that is:
      \begin{align*}
        f; (g; h) &= (f; g); h
      \end{align*}
      for all $f \in Hom (A, B)$, $g \in Hom (B, C)$, and $h \in Hom
      (C, D)$ and objects $A,B,C,D$.
    \end{itemize}
  \end{block}
\end{frame}
\begin{frame}
  \frametitle{Examples of categories (not small)}
  \begin{block}<+->{\textbf{Set}}
    Objects are sets and morphisms are total functions. 
  \end{block}
  \begin{block}<+->{\textbf{Par}}
    Objects are sets, morphisms are partial functions.
  \end{block}
  \begin{block}<+->{\textbf{Group}, \textbf{Ring}, \textbf{Vect}}
    Objects are groups (rings, vector spaces), morphisms are group
    (ring, vector space) homomorphisms
  \end{block}
\end{frame}
\begin{frame}
  \frametitle{Smaller categories}
  \begin{block}<+->{\textbf{FinSet} (only essentially small)}
    Objects are finite sets and morphisms are total functions.
  \end{block}
  \begin{block}<+->{Partially ordered sets}
    Every poset $(A, \le)$ gives rise to a category with objects $a\in
    A$ and a single morphism $m_{ab}$ for each $a, b\in A$ such that
    $a \le b$. 
  \end{block}
  \begin{block}<+->{Graphs}
    Every graph $(N, E)$ gives rise to category with objects $n\in N$
    and morphisms paths in $N$.
  \end{block}
  \begin{block}<+->{Hask (small?)}
    Objects are Haskell types, morphisms are Haskell functions.
  \end{block}
\end{frame}
\section{Functors}
\begin{frame}[fragile]
  \frametitle{Functors (on Hask)}
  \begin{block}<+->{Definition}
    A \alert{Functor} is a mapping \lstinline{f} between types such that for
    every pair of type \lstinline{a} and \lstinline|b| there is a
    function 
    \lstinline|fmap :: (a -> b) -> (f a -> f b)|
    such that the \alert{functorial laws} hold:
    \begin{enumerate}
    \item the identity function on \lstinline|a| is mapped to the
      identity function on \lstinline|f a|:\\
      \lstinline|fmap id fx == id fx|, \quad for all \lstinline|fx| in
      \lstinline|f a|
    \item \lstinline|fmap| is compatible with function composition\\
      \lstinline|fmap (f . g) == fmap f . fmap g|, \quad for all
      \lstinline|f :: b -> c| and \lstinline|g :: a -> b|
    \end{enumerate}
  \end{block}
  \begin{alertblock}<+->{Functions on types}
    \begin{itemize}
    \item \lstinline{Int}, \lstinline{Bool}, \lstinline{Double} etc
      are types.
    \item parameterized types like \lstinline$[a]$,
      \lstinline$BTree a$, \lstinline$IO a$ can be considered as a
      type constructor (i.e., \lstinline$[]$, \lstinline$BTree$,
      \lstinline$IO$) applied to a type 
    \item We can express that formally by writing \alert{kindings}:
      \lstinline{Int :: *},
      \lstinline{Bool :: *},
      \lstinline{Double :: *}, but
      \lstinline{[] :: * -> *},
      \lstinline{BTree :: * -> *},
      \lstinline{IO :: * -> *}
    \end{itemize}
  \end{alertblock}
\end{frame}

\begin{frame}[fragile]
  \frametitle{Functors in Haskell}
\begin{alertblock}<+->{The functor class}
\begin{lstlisting}
class Functor f where
  fmap :: (a -> b) -> (f a -> f b)
\end{lstlisting}
    \begin{itemize}
  \item Recall \texttt{f} is a type variable that can
    stand for \textbf{type
      constructors} (ie, functions on types) like  \texttt{IO},
    \texttt{[]}, and others.  So \lstinline|f :: * -> *|!
  \end{itemize}
\end{alertblock}
\begin{exampleblock}<+->{Good news}
  We already know a couple of functors!
\end{exampleblock}
\end{frame}

\begin{frame}
  \frametitle{List is a functor}
  \begin{itemize}[<+->]
  \item To make list an instance of functor, we need to instantiate
    the type \lstinline|f| in the type of \lstinline|fmap| by
    \lstinline|[]|, the list type constructor
  \item \lstinline|fmap :: (a -> b) -> (f a -> f b)|
  \item \lstinline|fmap :: (a -> b) -> ([a] -> [b])|
  \item Looks familiar?
  \item It's the type of \lstinline|map|
  \item It remains to check the functorial laws on \lstinline|map|
  \end{itemize}
\end{frame}
\begin{frame}[fragile]
  \frametitle{Functorial laws for list}
  \begin{block}<+->{\lstinline|fmap id fx == id fx|}
    \lstinline|fx| is a list, so we must proceed by induction
    \begin{itemize}
    \item \lstinline|map id [] == [] == id []|
    \item \lstinline|map id (x:xs) == id x : map id xs == x : xs == id (x : xs)|
    \end{itemize}
  \end{block}
  \begin{block}<+->{\lstinline$fmap (f . g) == fmap f . fmap g$}
    Must hold when applied to any list \lstinline$fx$
    \begin{itemize}
    \item \lstinline$map (f . g) [] == [] == map f (map g [])$
    \item \lstinline$map (f . g) (x : xs) == (f . g) x : map (f . g) xs$ \\
      \lstinline$== f (g x) : (map f . map g) xs$ by function
      composition and induction \\
      \lstinline$== f (g x) : map f (map g xs)$ by function composition
      \\
      \lstinline$== map f (g x : map g xs)$ by \lstinline$map f$
      \\
      \lstinline$== map f (map g (x : xs))$ by \lstinline$map g$
      \\
      \lstinline$== (map f . map g) (x : xs)$
    \end{itemize}
  \end{block}
\end{frame}
\begin{frame}
  \frametitle{Maybe is a functor}
  \begin{itemize}[<+->]
  \item Reminder: \lstinline$data Maybe a = Nothing | Just a$
  \item To make \lstinline$Maybe$ an instance of functor, we need to instantiate
    the type \lstinline$f$ in the type of \lstinline|fmap| by the type constructor
    \lstinline|Maybe|
  \item \lstinline|fmap :: (a -> b) -> (f a -> f b)|
  \item \lstinline|mapMaybe :: (a -> b) -> (Maybe a -> Maybe b)|
  \item There is actually no real choice for its definition
  \item \lstinline|mapMaybe g Nothing = Nothing|
  \item \lstinline|mapMaybe g (Just a) = Just (g a)| 
  \item Second equation could return \lstinline$Nothing$, but that
    would violate the functorial laws
  \item It remains to check the functorial laws on \lstinline|mapMaybe|
  \end{itemize}
\end{frame}
\begin{frame}[fragile]
  \frametitle{Functorial laws for Maybe}
  \begin{block}<+->{\lstinline$fmap id fx == id fx$}
    \lstinline$fx$ is a Maybe, so we must proceed by induction (cases)
    \begin{itemize}
    \item \lstinline$mapMaybe id Nothing == Nothing == id Nothing$
    \item \lstinline$mapMaybe id (Just x) == Just x == id (Just x)$
    \end{itemize}
  \end{block}
  \begin{block}<+->{\lstinline$fmap (f . g) == fmap f . fmap g$}
    Must hold when applied to any Maybe \lstinline$fx$
    \begin{itemize}
    \item \lstinline$mapMaybe (f . g) Nothing == Nothing == map f (map g Nothing)$
    \item \lstinline$mapMaybe (f . g) (Just x)$ \\
      \lstinline$== Just ((f . g) x)$ 
      \\
      \lstinline$== Just (f (g x))$ by function composition
      \\
      \lstinline$== mapMaybe f (Just (g x))$ by \lstinline$map f$
      \\
      \lstinline$== mapMaybe f (mapMaybe g (Just x))$ by \lstinline$map g$
      \\
      \lstinline$== (mapMaybe f . mapMaybe g) (Just x)$
    \end{itemize}
  \end{block}
\end{frame}

\begin{frame}[fragile]
  \frametitle{BTree is a functor}
  \begin{itemize}[<+->]
  \item Reminder:
    \lstinline$data BTree a = Leaf | Node (BTree a) a (BTree a)$
  \item To make \lstinline$BTree$ an instance of functor, we need to instantiate
    the type \lstinline$f$ in the type of \lstinline|fmap| by the type constructor
    \lstinline|BTree|
  \item \lstinline|fmap :: (a -> b) -> (f a -> f b)|
  \item \lstinline|mapBTree :: (a -> b) -> (BTree a -> BTree b)|
  \item There is actually no real choice for its definition
\begin{lstlisting}
mapBTree g Leaf = Leaf
mapBTree g (Node l a r) = Node (mapBTree g l) (g a) (mapBTree g r)
\end{lstlisting}
  \item In the second equation we need to transform the data at the
    node by \lstinline$g$ and the subtrees of type
    \lstinline$BTree a$ recursively to \lstinline$BTree b$ using the
    \lstinline$mapBTree$ function
  \item It remains to check the functorial laws on
    \lstinline|mapBTree|, but we'll leave this inductive proof to you.
  \end{itemize}
\end{frame}
\begin{frame}
  \frametitle{Remark}
  \begin{itemize}
  \item Many of the predefined type constructors have \texttt{Functor}
    instances
  \item Some of them may be unexpected
  \item For instance \lstinline|instance Functor ((,) a)| makes the pair
    type into a functor by defining \texttt{fmap} on the second
    component
  \item Mapping on the first component would also define a (different) functor!
  \end{itemize}
\end{frame}
\section{Applicatives}
\begin{frame}
  \frametitle{Applicatives}
  \begin{itemize}
  \item An applicative (functor) is a special kind of functor
  \item It has further operations and laws
  \item We motivate it with a couple of examples
  \end{itemize}
\end{frame}

\begin{frame}[fragile]
  \frametitle{Applicative}
  \begin{alertblock}<+->{Example 1: sequencing IO commands}
\begin{lstlisting}
sequence :: [IO a] -> IO [a]
sequence []       = return []
sequence (io:ios) = do x <- io
                       xs <- sequence ios
                       return (x:xs)
\end{lstlisting}  
  \end{alertblock}
  \begin{alertblock}<+->{Alternative way}
\begin{lstlisting}
sequence []       = return []
sequence (io:ios) = return (:) `ap` io `ap` sequence ios

return :: Monad m => a -> m a
ap     :: Monad m => m (a -> b) -> m a -> m b
\end{lstlisting}
  \end{alertblock}
\end{frame}

\begin{frame}[fragile]
 \frametitle{Applicative}
 \begin{exampleblock}<+->{Example 2: transposition}
\begin{lstlisting}
transpose :: [[a]] -> [[a]]
transpose [] = repeat []
transpose (xs:xss) = zipWith (:) xs (transpose xss)
\end{lstlisting}
 \end{exampleblock}
 \begin{alertblock}<+->{Rewrite}
\begin{lstlisting}
transpose []       = repeat []
transpose (xs:xss) = repeat (:) `zapp` xs `zapp` transpose xss

zapp :: [a -> b] -> [a] -> [b]
zapp fs xs = zipWith ($) fs xs
\end{lstlisting}
 \end{alertblock}
\end{frame}             

\begin{frame}[fragile]
  \frametitle{Applicative Interpreter}
  \begin{block}<+->{A datatype for expressions}
\begin{lstlisting}
data Exp v
  = Var v               -- variables
  | Val Int             -- constants
  | Add (Exp v) (Exp v) -- addition
\end{lstlisting}
  \end{block}
  \begin{exampleblock}<+->{Standard interpretation}
\begin{lstlisting}
eval :: Exp v -> Env v -> Int
eval (Var v) env = fetch v env
eval (Val i) env = i
eval (Add e1 e2) env = eval e1 env + eval e2 env

type Env v = v -> Int
fetch :: v -> Env v -> Int
fetch v env = env v
\end{lstlisting} 
\end{exampleblock}
\end{frame}

\begin{frame}[fragile]
  \frametitle{Applicative Interpreter}
  \begin{alertblock}{Alternative implementation}
\begin{lstlisting}
eval' :: Exp v -> Env v -> Int
eval' (Var v) = fetch v
eval' (Val i) = const i
eval' (Add e1 e2) = const (+) `ess` (eval' e1) `ess` (eval' e2)

ess a b c = (a c) (b c)
\end{lstlisting} 
\end{alertblock}
\end{frame}

\begin{frame}[fragile]
  \frametitle{Applicative}
\begin{exampleblock}{Extract the common structure}
\begin{lstlisting}
class Functor f => Applicative f where
  pure  :: a -> f a
  (<*>) :: f (a -> b) -> f a -> f b
\end{lstlisting} 
\end{exampleblock}
\end{frame}            

\begin{frame}[fragile]
  \frametitle{Applicative}
  \begin{block}{Laws}
  \begin{itemize}       
  \item Identity
\begin{lstlisting}
pure id <*> v == v
\end{lstlisting}
  \item Composition
\begin{lstlisting}
pure (.) <*> u <*> v <*> w = u <*> (v <*> w)
\end{lstlisting}
  \item Homomorphism
\begin{lstlisting}
pure f <*> pure x = pure (f x)
\end{lstlisting}
  \item Interchange
\begin{lstlisting}
u <*> pure y = pure ($ y) <*> u
\end{lstlisting}
  \end{itemize} 
  \end{block}   
\end{frame}

\begin{frame}[fragile]
  \frametitle{Instances of Applicative}
  \begin{itemize}
  \item<+-> List, Maybe, and IO are also applicatives
  \end{itemize}
  \begin{block}<+->{Lists}
\begin{lstlisting}
instance Applicative [] where
  -- pure :: a -> [a]
  pure a = [a]
  -- (<*>) :: [a -> b] -> [a] -> [b]
  fs <*> xs = concatMap (\f -> map f xs) fs
\end{lstlisting}
  \end{block}
  \begin{block}<+->{Maybe}
\begin{lstlisting}
instance Applicative Maybe where
  -- pure :: a -> Maybe a
  pure a = Just a
  -- (<*>) :: Maybe (a -> b) -> Maybe a -> Maybe b
  Just f <*> Just a = Just (f a)
  _ <*> _ = Nothing
\end{lstlisting}
  \end{block}
\end{frame}

\section{Parsers}
\begin{frame}[fragile]
  \frametitle{An interesting example for Applicatives}
  \begin{block}<+->{Simple arithmetic expressions}
\begin{lstlisting}
data Term  = Con Integer
           | Bin Term Op Term  
             deriving (Eq, Show)
           
data Op    = Add | Sub | Mul | Div
             deriving (Eq, Show)
\end{lstlisting}
\end{block}
\begin{alertblock}<+->{Parsing expressions}
  \begin{itemize}
  \item Read a string like \texttt{"3+42/6"}
  \item Recognize it as a valid term
  \item Return \texttt{Bin (Con 3) Add (Bin (Con 42) Div (Con 6))} 
  \end{itemize}
\end{alertblock}
\end{frame}
\begin{frame}[fragile]
  \frametitle{Parsing}
\begin{block}{The type of a simple parser}
\begin{lstlisting}
type Parser token result = [token] -> [(result, [token])]
\end{lstlisting}  
\end{block}
\end{frame}             

\begin{frame}[fragile]
  \frametitle{Combinator parsing}
  \begin{block}{Primitive parsers}
\begin{lstlisting}
pempty :: Parser t r
succeed :: r -> Parser t r
satisfy :: (t -> Bool) -> Parser t t
msatisfy :: (t -> Maybe a) -> Parser t a
lit :: Eq t => t -> Parser t t
\end{lstlisting}
  \end{block}
\end{frame}

\begin{frame}[fragile]
  \frametitle{Combinator parsing II}
  \begin{block}{Combination of parsers}
\begin{lstlisting}
palt :: Parser t r -> Parser t r -> Parser t r
pseq :: Parser t (s -> r) -> Parser t s -> Parser t r
pmap :: (s -> r) -> Parser t s -> Parser t r
\end{lstlisting}
\end{block}     
\end{frame}             


\begin{frame}[fragile]
  \frametitle{A taste of compiler construction}
\begin{block}{A lexer}
 A lexer partitions the incoming list of
 characters into a list of tokens. A token is either a single symbol, 
 an identifier, or a number. Whitespace characters are removed.
\end{block}
\end{frame}     


\begin{frame}[fragile]
  \frametitle{Underlying concepts}
  \begin{alertblock}{Parsers have a rich structure}
    \begin{itemize}
    \item parsing illustrates functors, applicatives, as well as
      monads that we already saw in the guise of IO instructions
    \end{itemize}
  \end{alertblock}
\end{frame}             

\begin{frame}[fragile]
  \frametitle{Parsing is \dots}
  \begin{block}{A functor}
    Check the functorial laws!
  \end{block}
  \begin{block}{An applicative}
    Check applicative laws!
  \end{block}
  \begin{block}{A monad}
    Check the monad laws (upcoming)!
  \end{block}
  \begin{alertblock}{Consequence}
    Can use \texttt{do} notation for parsing!
  \end{alertblock}
\end{frame}

\begin{frame}[fragile]
  \frametitle{Parsers are Applicative!}
  \begin{alertblock}{}
\begin{lstlisting}
instance Applicative (Parser' token) where
  pure = return
  (<*>) = ap

instance Alternative (Parser' token) where
  empty = mzero
  (<|>) = mplus
\end{lstlisting}
  \end{alertblock}
\end{frame}

\begin{frame}
  \frametitle{Wrapup}
  \begin{itemize}[<+->]
  \item what if there are multiple applicatives?
  \item they just compose (unlike monads)
  \item applicative do notation
  \item applicatives cannot express dependency
  \item enable more clever parsers
  \end{itemize}
  
\end{frame}

\end{document}
