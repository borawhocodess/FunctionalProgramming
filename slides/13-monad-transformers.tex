%% -*- coding: utf-8 -*-
\documentclass[pdftex,aspectratio=169]{beamer}

\usepackage[T1]{fontenc}
\usepackage[utf8]{inputenc}
\usepackage{beramono}

\input{common}
%%% frontmatter
%% -*- coding: utf-8 -*-

\title{Functional Programming}
\subtitle{Introduction}

\author[Peter Thiemann]{Prof. Dr. Peter Thiemann}
\institute[Univ. Freiburg]{Albert-Ludwigs-Universität Freiburg, Germany}
\date{WS 2024/25}


\author[Gabriel Radanne]{Dr. Gabriel Radanne \and Peter Thiemann}
\subtitle
{Monad Transformers}


\renewcommand\CodeFont{\ttfamily}
\lstset{%
  frame=none,
  language=Haskell
}

\begin{document}

\begin{frame}
  \titlepage
\end{frame}

\begin{frame}
  \frametitle{Reminder: Monad}
  \begin{block}{Definition of a Monad -- Previous lecture}
    \begin{itemize}
    \item abstract datatype for instructions that produce values
    \item built-in combination \lstinline{>>=}
    \item abstracts over different interpretations (computations)
    \end{itemize}
  \end{block}
\end{frame}

\begin{frame}[fragile]
  \frametitle{Monad definition}
  \begin{block}{The type class Monad}
\begin{lstlisting}
class Monad m where
  (>>=)  :: m a -> (a -> m b) -> m b
  return :: a -> m a
  fail   :: String -> m a
\end{lstlisting}

with the following laws:
\begin{itemize}
\item \lstinline{return x >>= f == f x}
\item \lstinline{m >>= return == m}
\item \lstinline{(m >>= f) >>= g == m >>= (\x -> f x >>= g)}
\end{itemize}

\end{block}
\end{frame}

\begin{frame}[fragile]
  \frametitle{What about Composition?}
  \begin{itemize}[<+->]
  \item What does it even mean?
  \item Given two \textbf{functors} \lstinline{f} and \lstinline{g},
    is \lstinline{f . g} also a functor?\\
    I.e., the type constructor
    that first applies \lstinline{g} and then \lstinline{f}?
  \item Can make that more formal
\begin{lstlisting}
newtype Comp f g a = Comp (f (g a))
\end{lstlisting}
  \item The type constructor \lstinline{Comp} has an interesting
    kinding that corresponds to the type of function composition:
\begin{lstlisting}
Comp :: (* -> *) -> (* -> *) -> (* -> *)
\end{lstlisting}
  \item Questions
    \begin{itemize}
    \item If \lstinline{f} and \lstinline{g} are functors, what about
      \lstinline{Comp f g}?
    \item If \lstinline{f} and \lstinline{g} are applicatives, what about
      \lstinline{Comp f g}?
    \item If \lstinline{f} and \lstinline{g} are monads, what about
      \lstinline{Comp f g}?
    \end{itemize}
  \item We sometimes want to use multiple functors, applicatives, monads at once! 
  \end{itemize}
\end{frame}

\begin{frame}
  \frametitle{Why combine functors, applicatives, monads}
  Lecture 12: Monadic interpreters.

  Interpreters can have many features:
  \begin{itemize}
  \item Failure (\lstinline{Maybe}).
  \item Keeping some state (\lstinline{State}).
  \item Reading from the environment (\lstinline{Reader}).
  \item \dots
  \end{itemize}

  To implement an interpreter, we need to combine all these monads!
\end{frame}

\begin{frame}
  \frametitle{Let's start by combining functors!}
  \begin{block}{To show that \lstinline{Comp f g} is a functor \dots}
    \begin{itemize}
    \item Implement \lstinline{fmap} (i.e., give an instance of the
      \lstinline{Functor} class)
    \item Show that the functor laws hold
      \begin{itemize}
      \item The identity function gets mapped to the identity
        function.
      \item Functor composition commutes with function composition.
      \end{itemize}
    \end{itemize}
  \end{block}
\end{frame}

\begin{frame}
  \frametitle{Let's combine applicatives!}
  \begin{block}{To show that \lstinline{Comp f g} is an applicative \dots}
    \begin{itemize}
    \item Implement \lstinline{pure} and \lstinline{(<*>)} (i.e., give an instance of the
      \lstinline{Applicative} class)
    \item Show that the applicative laws hold \dots
    \end{itemize}
  \end{block}

\end{frame}

\begin{frame}[fragile]
  \frametitle{Let's combine Monads! -- State alone}

  \begin{block}{The State monad}
\begin{lstlisting}
data ST s a = ST { runST :: s -> (s, a) }

instance Functor (ST s) where
  fmap h sg = ST (fmap h . runST sg)

instance Applicative (ST s) where
  pure a = ST (, a)
  ST fab <*> ST fa = ST $ \s -> let (s', f) = fab s in
                                fmap f $ fa s'

instance Monad (ST s) where
  ST fa >>= h = ST $ \s -> let (s', a) = fa s in
                           runST (h a) s'
\end{lstlisting}
  \end{block}
\end{frame}

\begin{frame}[fragile]
  \frametitle{Let's combine Monads! - Maybe+State}
  \begin{block}<+->{Consider  \lstinline{Comp (ST s) Maybe}}
    \begin{itemize}
    \item Corresponds to \lstinline{s -> (s, Maybe a)}, a stateful
      computation that may fail
    \item \lstinline{return} and \lstinline{>>=} are easy to define
    \end{itemize}
  \end{block}
  \begin{block}<+->{Consider \lstinline{Comp Maybe (ST s)}}
  \begin{itemize}
  \item Corresponds to \lstinline{Maybe (s -> (s, a))} 
    \item \lstinline{return a = return_Maybe (return_ST a)}
    \item But there's not way to write the bind function:
\begin{lstlisting}
Nothing  >>= f = Nothing
Just sta >>= f = ???
\end{lstlisting}
  \end{itemize}
\end{block}
\begin{block}<+->{Lessons}
  \begin{itemize}
  \item Monads do not compose, in general
  \item Monad composition is not commutative!
  \end{itemize}
\end{block}
\end{frame}


\begin{frame}[fragile]
  \frametitle{Let's combine Monads! -- Maybe+State}
  \begin{block}<+->{A different construction: The MaybeState monad}
    \begin{itemize}
    \item Purpose: propagate state and signaling of errors
    \item Attention: the state is lost
    \end{itemize}
      \begin{lstlisting}
data MaybeState s a = MST { runMST :: s -> Maybe (s, a) }

....

instance Monad (MST s) where
  return a = MST (\s -> return (s, a))
  ms >>= f = MST (\s -> runMST ms s >>= \(s', a) -> runMST (f a) s')
      \end{lstlisting}
  \end{block}
  \begin{itemize}
  \item<+->     Interestingly, the implementation does not depend 
    on \lstinline{Maybe} at all! 
  \item<+->     We don't have to write this definition again for other combinations! 
  \end{itemize}
\end{frame}

\begin{frame}[fragile]
  \frametitle{Alternative solution: Monad transformers}

  We've seen a particular instance of a \emph{monad transformer}:
  \begin{block}<+->{}
\begin{lstlisting}
class MonadTrans t where
  lift :: Monad m => m a -> t m a
\end{lstlisting}

\end{block}

A monad transformer \lstinline{t} takes a monad \lstinline{m} and yields a new monad
\lstinline{(t m)}.

Function \lstinline{lift} moves a computation from the underlying
monad to the new monad.

\begin{block}<+->{Intermezzo}
  \begin{itemize}[<+->]
  \item What's the kind of \lstinline{t} in \lstinline{MonadTrans}?
  \item Answer: \lstinline{t :: (* -> *) -> (* -> *)}
  \end{itemize}
\end{block}
\end{frame}

\begin{frame}[fragile]
  \frametitle{The \texttt{MaybeT} monad transformer}
  \begin{block}{Definition}
\begin{lstlisting}
newtype MaybeT m a = MaybeT { runMaybeT :: m (Maybe a) }

instance (Monad m) => Monad (MaybeT m) where
  return = MaybeT . return . Just
  mmx >>= f = MaybeT $ do
    mx <- runMaybeT mmx
    case mx of
      Nothing -> return Nothing
      Just x  -> runMaybeT (f x)

instance MonadTrans MaybeT where
  lift mx = MaybeT $ mx >>= (return . Just)
\end{lstlisting}
  \end{block}
\end{frame}

\begin{frame}[fragile]
  \frametitle{A simple use of MaybeT}
  We can recover the ``normal'' monad by applying to \lstinline{Identity}.
  \begin{block}{}
  \begin{lstlisting}
type MaybeLike = MaybeT Identity
  \end{lstlisting}
\end{block}
\end{frame}


\begin{frame}[fragile]
  \frametitle{The \texttt{StateT} monad transformer}
  \begin{block}{Definition}
\begin{lstlisting}
newtype StateT s m a = StateT { runStateT :: s -> m (s, a) }

instance (Monad m) => Monad (StateT s m) where
  return a = StateT $ \s -> return (s, a)
  m >>= f  = StateT $ \s -> do
      (s', a) <- runStateT m s
      runStateT (f a) s'

instance MonadTrans (StateT s) where
  lift ma = StateT $ \s -> do { a <- ma ; return (s, a) }
\end{lstlisting}
  \end{block}
\end{frame}


\begin{frame}[fragile]
  \frametitle{Let's combine Monads with transformers!}

  Demo!
\end{frame}


\begin{frame}[fragile]
  \frametitle{The \texttt{ReaderT} monad transformer}
  \begin{block}{Definition}
\begin{lstlisting}
newtype ReaderT r m a = ReaderT { runReaderT :: r -> m a }

ask :: (Monad m) => ReaderT r m r
ask = ReaderT return

instance Monad m => Monad (ReaderT r m) where
    return  = lift . return
    m >>= k = ReaderT $ \r -> do
                 a <- runReaderT m r
                 runReaderT (k a) r

instance MonadTrans (ReaderT r) where
    lift m = ReaderT (const m)
\end{lstlisting}
  \end{block}
\end{frame}

\begin{frame}[fragile]
  \frametitle{Back to interpreters}

  \begin{block}{Earlier we had a monadic interpreter for:}
    \begin{lstlisting}
data Term  = Con Integer
           | Bin Term Op Term
             deriving (Eq, Show)

data Op    = Add | Sub | Mul | Div
             deriving (Eq, Show)
    \end{lstlisting}
  \end{block}
  \pause
  Different interpreters with various features:
  \begin{itemize}
  \item Failure ($\Rightarrow$ exception/Maybe monad)
  \item Counting instructions ($\Rightarrow$ state monad)
  \item Traces ($\Rightarrow$ writer monad)
  \end{itemize}
\end{frame}


\begin{frame}
  \frametitle{Key points}
  \begin{itemize}[<+->]
  \item Monads do not always compose:\\ if \lstinline{m1} and
    \lstinline{m2} are monads, 
    there is no general definition that makes \lstinline{Comp m1 m2} a
    monad 
  \item But monad transformers help.
  \item Order is important:\\
    \lstinline{StateT s Maybe} $\ne$ \lstinline{MaybeT (ST s)}
  \item You should not overdo it.
  \item It's all in the \lstinline{mtl} library.
  \end{itemize}
\end{frame}

\end{document}




%%% Local Variables:
%%% mode: latex
%%% TeX-master: t
%%% End:
